\documentclass{article}
\usepackage[utf8]{inputenc}
\usepackage{graphicx}
\usepackage{booktabs}
\usepackage{gensymb}
\usepackage{float}
\usepackage{color}
\usepackage{amssymb}
\usepackage{mathtools}
\usepackage{mathpartir}
\usepackage{listings}
\usepackage[spanish,es-tabla]{babel}
\title{Trabajo Práctico Final}
\author{Integrantes: Tomás Fernández De Luco y Ignacio Sebastián Moliné}
\date{28 de septiembre de 2018}
\topmargin=-2cm
\oddsidemargin=0cm
\textheight=24cm
\textwidth=17cm
\newcommand*\rfrac[2]{{}^{#1}\!/_{#2}}
\begin{document}
\begin{titlepage}

\begin{minipage}{2.6cm}
\includegraphics[width=\textwidth]{fceia.pdf}
\end{minipage}
\hfill
%
\begin{minipage}{6cm}
\begin{center}
\normalsize{Universidad Nacional de Rosario\\
Facultad de Ciencias Exactas,\\
Ingeniería y Agrimensura\\}
\vspace{0.5cm}
\large
Lic. en Cs. de la Computación
\end{center}
\end{minipage}
\hspace{0.5cm}
\hfill
\begin{minipage}{2.6cm}
\includegraphics[width=\textwidth]{unr.pdf}
\end{minipage}

\vspace{5.5cm}

\begin{center}
\LARGE{\sc Análisis de Lenguajes de Programación}\\
\vspace{0.5cm}
\large{Trabajo Práctico 1}\\

\vspace{5cm}

\large
Tomás Fernández De Luco F-3443/6\\
Ignacio Sebastián Moline M-6466/1\\

\vspace*{0.5cm}
\small{28 de septiembre de 2018}

\makeatletter
\def\@seccntformat#1{%
  \expandafter\ifx\csname c@#1\endcsname\c@section\else
  \csname the#1\endcsname\quad
  \fi}
\makeatother

\lstdefinestyle{myCustom}{
	language=Haskell,
	numbersep=10pt,
	tabsize=2,
	showspaces=false,
	showstringspaces=false,
	keepspaces=true,
	frame=single,
	commentstyle=\color{blue}
}


\end{center}
\end{titlepage}
\lstset{basicstyle=\small,style=myCustom}
	\newpage
	
	\section*{Ejercicio 1}
	El operador '?:' evalúa a la primera expresión entera si la condición booleana es verdadera
	y a la segunda en caso contrario. Por lo tanto, para representarlo en la sintaxis se necesitará 
	una boolexp seguida de dos intexp. La sintaxis abstracta modificada para intexp será: \\
\\
	$\begin{aligned}
		intexp ::=\ &nat\ |\ var\ |\ -_{u} intexp\\
		|\ &intexp + intexp\\
		|\ &intexp -_{b} intexp\\
		|\ &intexp \times intexp\\
		|\ &intexp \div intexp\\
		|\ &boolexp\ ?\ intexp : intexp
	\end{aligned}\\$
\\
\\
	Y la sintaxis concreta correspondiente será:\\
	$\begin{aligned}
\\	intexp ::=\ &nat\\
	|\ &var\\
	|\ &\text{'-'}\ intexp\\
	|\ &intexp\ \text{'+'}\ intexp\\
	|\ &intexp\ \text{'-'}\ intexp\\
	|\ &intexp\ \text{'*'}\ intexp\\
	|\ &intexp\ \text{'/'}\  intexp\\
	|\ &\text{'('}\ intexp\ \text{')'}\\
	|\ &boolexp\ \text{'?'}\ intexp\ \text{':'}\ intexp
	\end{aligned}$
	\section*{Ejercicio 2}
	Se debe agregar un constructor más al tipo de IntExp con los tipos de datos mencionados previamente.
	\lstinputlisting[language=Haskell, firstline=6, lastline=15]{AST.hs}
	\newpage
	\section*{Ejercicio 3}
	El código del archivo Parser.hs es el siguiente:
	\lstinputlisting[language=Haskell, firstline=1, lastline=130]{Parser.hs}
	
	\section*{Ejercicio 4}
	Se necesitarán dos reglas para representar el comportamiento esperado por el operador ternario '?:':
	\[
		\inferrule* [right=TernTrue]
		{\inferrule* [] {}{\langle p,\sigma \rangle \Downarrow _{boolexp} $ \textbf{true}$}
	\\
	\inferrule* []{}{\langle e_{1},\sigma \rangle \Downarrow _{intexp} n}}
	{\langle p$ ? $e_{1}:e_{2},\sigma \rangle \Downarrow _{intexp} n}
	\] \\
	\[
		\inferrule* [right=TernFalse]
		{\inferrule* [] {}{\langle p,\sigma \rangle \Downarrow _{boolexp} $ \textbf{false}$}
	\\
	\inferrule* []{}{\langle e_{2},\sigma \rangle \Downarrow _{intexp} n}}
	{\langle p$ ? $e_{1}:e_{2},\sigma \rangle \Downarrow _{intexp} n}
	\]
	
	\section*{Ejercicio 5}
	
Para demostrar el determinismo de la relación, se probará que si $t \leadsto t'$ y $t\leadsto t''$, entonces $t' = t''$, mediante inducción en la derivación $t \leadsto t'$. Para ello, se supondrá que las relaciones de evaluación big-step para expresiones enteras y booleanas son deterministas.\\

Si la última regla utilizada es ASS, t = $\langle v:=e, \sigma\rangle$ y t\textsc{\char13} = $\langle skip, [\sigma | v : n]\rangle$, sabiendo que $\langle e, \sigma\rangle \Downarrow_{intexp} n.$ 
Luego, en $t\leadsto t\textsc{\char13}\textsc{\char13}$ no se puede haber aplicado como última regla SEQ$_{1}$, SEQ$_{2}$, IF$_{1}$, IF$_{2}$ o REPEAT, pues ellas requieren que el comando sea de un tipo distinto a la asignación. Por lo tanto, la última regla aplicada en $t\leadsto t\textsc{\char13}\textsc{\char13}$ debe ser ASS. Como la evaluación de expresiones enteras es determinista, tenemos que si $\langle e, \sigma\rangle \Downarrow_{intexp} n$ y $\langle e, \sigma\rangle \Downarrow_{intexp} n\textsc{\char13}$, luego $n = n\textsc{\char13}$. Por lo tanto, $t\textsc{\char13}\textsc{\char13} = \langle skip, [\sigma | v : n]\rangle = t\textsc{\char13}. $\\

Si la última regla fue SEQ$_{1}$, t = $\langle skip;c_{1}, \sigma\rangle$ y $t\textsc{\char13} = \langle c_{1}, \sigma\rangle$. Entonces, en $t \leadsto t\textsc{\char13}\textsc{\char13}$ no se puede haber aplicado ASS, IF$_{1}$, IF$_{2}$ o REPEAT, pues ellas requieren que el comando sea distinto de skip. Tampoco puede haber sido SEQ$_{2}$ ya que requeriría que $\langle skip, \sigma\rangle \leadsto \langle c\textsc{\char13}, \sigma \textsc{\char13}\rangle$, pero no existen reglas que permitan derivarlo, pues es un valor. Por lo tanto, la última regla aplicada en $t \leadsto t\textsc{\char13}\textsc{\char13}$ debe ser SEQ$_{1}$. Entonces, $t\textsc{\char13}\textsc{\char13} = \langle c_{1}, \sigma\rangle = t\textsc{\char13}.$\\

Si la última regla fue SEQ$_{2}$, t = $\langle c_{0};c_{1}, \sigma\rangle$ y $t\textsc{\char13} = \langle c_{0}\textsc{\char13};c_{1}, \sigma\textsc{\char13}\rangle$, sabiendo que $\langle c_{0}, \sigma\rangle \leadsto \langle c_{0}\textsc{\char13}, \sigma\textsc{\char13}\rangle$. Entonces, en $t \leadsto t\textsc{\char13}\textsc{\char13}$ no se puede haber aplicado ASS, IF$_{1}$, IF$_{2}$ o REPEAT, pues ellas requieren que el comando no sea una sucesión de dos subcomandos. Tampoco puede haber sido SEQ$_{1}$ ya que $\langle c_{0}, \sigma\rangle$ evalúa en un paso a $\langle c_{0}\textsc{\char13}, \sigma\textsc{\char13}\rangle,$ por la premisa de SEQ$_{2}$ en $t \leadsto t\textsc{\char13}$. Entonces, $c_{1}$ no puede ser skip, porque en ese caso, t no evaluaría a nada. Por lo tanto, la última regla aplicada en $t \leadsto t\textsc{\char13}\textsc{\char13}$ debe ser SEQ$_{2}$. Sabemos entonces que $t\textsc{\char13}\textsc{\char13} = \langle c_{0}\textsc{\char13}\textsc{\char13}; c_{1}, \sigma\textsc{\char13}\textsc{\char13}\rangle$, con $\langle c_{0}, \sigma\rangle \leadsto \langle c_{0}\textsc{\char13}\textsc{\char13}, \sigma\textsc{\char13}\textsc{\char13}\rangle$. Luego, por hipótesis inductiva, se tiene que la evaluación de $\langle c_{0}, \sigma\rangle$ es determinista, por lo que tenemos que $\langle c_{0}\textsc{\char13}, \sigma\textsc{\char13}\rangle = \langle c_{0}\textsc{\char13}\textsc{\char13}, \sigma\textsc{\char13}\textsc{\char13}\rangle$. Entonces, $t\textsc{\char13} = \langle c_{0}\textsc{\char13}; c_{1}, \sigma\textsc{\char13}\rangle = \langle c_{0}\textsc{\char13}\textsc{\char13}; c_{1}, \sigma\textsc{\char13}\textsc{\char13}\rangle = t\textsc{\char13}\textsc{\char13}$.\\

Si la última regla utilizada es IF$_{1}$, t = $\langle $if $ b$ then $c_{0}$ else $c_{1}, \sigma\rangle$ y $t\textsc{\char13} = \langle c_{0}, \sigma\rangle$, sabiendo que $\langle b, \sigma\rangle \Downarrow_{boolexp} true$. Luego, la última regla aplicada en $t\leadsto t\textsc{\char13}\textsc{\char13}$ no puede haber sido ASS, SEQ$_{1}$, SEQ$_{2}$ o REPEAT, pues ellas requieren que el comando sea de un tipo distinto a un if. Además, sabiendo que la relación de evaluación de expresiones booleanas es determinista, $\langle b, \sigma\rangle$ evalúa sólo a true, por lo que tampoco podría haberse usado IF$_{2}$. Por lo tanto, la última regla aplicada en $t\leadsto t\textsc{\char13}\textsc{\char13}$ debe ser IF$_{1}$. Entonces, tenemos que $t\textsc{\char13}\textsc{\char13} = \langle c_{0}, \sigma\rangle = t\textsc{\char13}$. La demostración es análoga para el caso de IF$_{2}$.\\

Si la última regla fue REPEAT, t = $\langle repeat $  c $ until $ b, $\sigma \rangle$ y $t\textsc{\char13} = \langle c;$ if $b$ then $skip$ else $repeat$ c $until$ b, $\sigma\rangle$. Entonces, en $t \leadsto t\textsc{\char13}\textsc{\char13}$ no se puede haber aplicado ninguna otra regla, pues necesitan comandos distintos a un repeat. Por lo tanto, la última regla aplicada en $t \leadsto t\textsc{\char13}\textsc{\char13}$ debe ser REPEAT. Entonces, $t\textsc{\char13}\textsc{\char13} = \langle c;$ if $b$ then $skip$ else $repeat$ c $until$ b, $\sigma\rangle = t\textsc{\char13}.$

	\newpage
	\section*{Ejercicio 6}
	Para la resolución del ejercicio, se optó por dividir el árbol de derivación, dado que es demasiado grande para mostrarlo completo, y se considera que su desarrollo y explicación serán más claros de esta manera. Se comenzará con desarrollar la siguiente expresión: 
	\[
	\inferrule* [right=SEQ$_{2}$]
	{\inferrule* [right=ASS] 
		{\inferrule* [right=PLUS]
			{\inferrule* [right=VAR]
				{\inferrule* [] {}{ }
					\\
				}
				{\langle x,[\sigma | x:0] \rangle \Downarrow _{intexp} 0 }
				\inferrule* [right=NVAL]
				{\inferrule* [] {}{ }
					\\
				}
				{\langle 1,[\sigma | x:0] \rangle \Downarrow _{intexp} 1}
			}
			{\langle x+1,[\sigma | x:0] \rangle \Downarrow _{intexp} 0+1 = 1}
		}
		{\langle x:=x+1, [\sigma | x:0] \rangle 
			\leadsto \langle skip, [\sigma | x:1] \rangle }
		\\
	}
	{\langle x:=x+1; $ if $ x>0$ then $skip$ else $x:=x-1, [\sigma | x:0] \rangle 
		\leadsto \langle skip; $ if $ x>0$ then $skip$ else $ x:=x-1, [\sigma | x:1] \rangle }
	\]\\

 Sean $t_{1} = \langle x:=x+1; $ if $ x>0$ then $skip$ else $x:=x-1, [\sigma | x:0] \rangle$ y $t_{2} = \langle skip; $ if $ x>0$ then $skip$ else $ x:=x-1, [\sigma | x:1] \rangle$. 
Puede verse por el árbol anterior que $t_{1} \leadsto t_{2}$. Luego, 
	
		\[
	\inferrule* [right=SEQ$_{1}$]
	{\inferrule* [] {}{ }}
	{t_{2} \leadsto \langle $if $ x>0$ then $skip$ else $ x:=x-1, [\sigma | x:1] \rangle}
	\]
	
Ahora, sea $t_{3} = \langle $if $ x>0$ then $skip$ else $ x:=x-1, [\sigma | x:1] \rangle $. Entonces, 

\[
\inferrule* [right=IF$_{1}$]
{\inferrule* [right=GT] 
	{\inferrule* [right=VAR]
		{\inferrule* [] {}{ }
			\\
		}
		{\langle x,[\sigma | x:1] \rangle \Downarrow _{intexp} 1 }
		\inferrule* [right=NVAL]
		{\inferrule* [] {}{ }
			\\
		}
		{\langle 0,[\sigma | x:0] \rangle \Downarrow _{intexp} 0}
	}
	{\langle x>0, [\sigma | x:1] \rangle \Downarrow _{boolexp} 1>0 = true}
	\\
}
{t_{3} \leadsto \langle skip, [\sigma | x:1] \rangle }
\]

Donde $t_{4} =\langle skip, [\sigma | x:1] \rangle$.

Como $\leadsto*$ es una clausura de $\leadsto$, tenemos que
\[
\inferrule* []
{t_{1} \leadsto  t_{2}}
{t_{1} \leadsto$* $  t_{2} }
\]
\[
\inferrule* []
{t_{2} \leadsto  t_{3}}
{t_{2} \leadsto$* $  t_{3} }
\]
\[
\inferrule* []
{t_{3} \leadsto  t_{4}}
{t_{3} \leadsto$* $  t_{4} }
\]

Por último, como la relación es transitiva, se tiene que:

\[
\inferrule* []
{\inferrule* []
	{t_{1} \leadsto$* $  t_{2} \quad t_{2} \leadsto$* $  t_{3}
	}
	{t_{1} \leadsto$* $  t_{3}
	}
	{t_{3} \leadsto$* $  t_{4} }
}
{t_{1} \leadsto$* $  t_{4} }
\]

 Completando así la prueba.

\newpage
	\section*{Ejercicio 7}
	Para el caso del error de división por cero, se deja que Haskell maneje el error y aborte la ejecución.
	Para las variables no definidas, se hizo un pattern matching no exhaustivo en la función $lookfor$ así no habrá un comportamiento especificado en caso de no encontrar la variable en la lista de estados. El código del archivo Eval1.hs es el siguiente:
	\lstinputlisting[language=Haskell, firstline=1, lastline=80]{Eval1.hs}
\newpage	
	\section*{Ejercicio 8}
	El código del archivo Eval2.hs es el siguiente:
	\lstinputlisting[language=Haskell, firstline=1, lastline=121]{Eval2.hs}
\newpage	
	\section*{Ejercicio 9}
	Ahora $evalComm$ devuelve un par cuyo primer elemento es un estado o error, y el segundo es la traza de asignaciones de variables hasta el final normal de la ejecución, o el error que la interrumpa. El código del archivo Eval3.hs es el siguiente:
	\lstinputlisting[language=Haskell, firstline=1, lastline=128]{Eval3.hs}
	\section*{Ejercicio 10}
	De manera similar al ejercicio 1, se extiende la sintaxis abstracta del tipo $comm$ para agregar un comando adicional. El resultado final será:\\
\\	$\begin{aligned}
		comm ::=\ &\textbf{skip}\\
		|\ &var\ :=\ intexp\\
		|\ &comm;comm\\
		|\ &\textbf{if\ } boolexp \textbf{\ then\  } comm \textbf{\ else\ } comm\\
		|\ &\textbf{repeat\ }comm\ boolexp\\
		|\ &\textbf{while\ } boolexp\  comm
	\end{aligned}\\$
\\
	El esquema de reglas para la sintaxis operacional de $while$ será muy similar a la del $repeat$. En caso de que la condición sea cierta, se realizará el comando seguido de volver a ejecutar el ciclo; si no, se terminará el comando con un $skip$. 
	\[
		\inferrule* [right=WHILE]
		{\inferrule* [] {}{ }
			\\
		}
		{\langle $while $b$ do $c, \sigma \rangle \leadsto \langle  $if $ b$ then $c; $ while $b $ do $c$ else $ skip, \sigma \rangle}
	\]
\end{document}
