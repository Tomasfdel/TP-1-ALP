\documentclass{article}
\usepackage[utf8]{inputenc}
\usepackage{graphicx}
\usepackage{booktabs}
\usepackage{gensymb}
\usepackage{float}
\usepackage{color}
\usepackage{amssymb}
\usepackage{mathtools}
\usepackage{mathpartir}
\usepackage{listings}
\usepackage[spanish,es-tabla]{babel}
\title{Trabajo Práctico Final}
\author{Integrantes: Tomás Fernández De Luco y Ignacio Sebastián Moliné}
\date{30 de noviembre de 2018}
\topmargin=-2cm
\oddsidemargin=0cm
\textheight=24cm
\textwidth=17cm
\newcommand*\rfrac[2]{{}^{#1}\!/_{#2}}
\begin{document}
\begin{titlepage}

\begin{minipage}{2.6cm}
\includegraphics[width=\textwidth]{fceia.pdf}
\end{minipage}
\hfill
%
\begin{minipage}{6cm}
\begin{center}
\normalsize{Universidad Nacional de Rosario\\
Facultad de Ciencias Exactas,\\
Ingeniería y Agrimensura\\}
\vspace{0.5cm}
\large
Lic. en Cs. de la Computación
\end{center}
\end{minipage}
\hspace{0.5cm}
\hfill
\begin{minipage}{2.6cm}
\includegraphics[width=\textwidth]{unr.pdf}
\end{minipage}

\vspace{5.5cm}

\begin{center}
\LARGE{\sc Análisis de Lenguajes de Programación}\\
\vspace{0.5cm}
\large{Trabajo Práctico 4}\\

\vspace{5cm}

\large
Tomás Fernández De Luco F-3443/6\\
Ignacio Sebastián Moline M-6466/1\\

\vspace*{0.5cm}
\small{30 de noviembre de 2018}

\makeatletter
\def\@seccntformat#1{%
  \expandafter\ifx\csname c@#1\endcsname\c@section\else
  \csname the#1\endcsname\quad
  \fi}
\makeatother

\lstdefinestyle{myCustom}{
	language=Haskell,
	numbersep=10pt,
	tabsize=2,
	showspaces=false,
	showstringspaces=false,
	keepspaces=true,
	frame=single,
	commentstyle=\color{blue}
}


\end{center}
\end{titlepage}
\lstset{basicstyle=\small,style=myCustom}
\newpage
\section*{Ejercicio 1}
\paragraph{} 
a) Demostración de State como mónada:
\begin{lstlisting}
newtype State a = State {runState :: Env -> (a, Env)}
 
instance Monad State where
    return x = State (\s -> (x, s))
    m >>= f = State (\s -> let (v, s') = runState m s
                           in runState (f v) s')
 
-- Monad 1
return y >>= f =
-- {Definicion de return}
State (\s -> (y, s)) >>= f =
-- {Definicion de >>=}
State (\s -> let (v, s') = runState (State (\s -> (y, s))) s
              in runState (f v) s') =                          
-- {Definicion de runState}
State (\s -> let (v, s') = (\s -> (y, s)) s
              in runState (f v) s') =                          
-- {Aplicacion de funcion anonima}
State (\s -> let (v, s') = (y, s)
              in runState (f v) s') =                          
-- {Sustitucion de let}
State (\s -> runState (f y) s) =                               
-- {Eta-reduccion}
State (runState (f y)) =                                       
-- {Sea x = State z. Luego, State (runState x) = State z = x}
-- {Entonces, la composicion de State y runState es la identidad}
f y

-- Monad 2
m >>= return =                                                 
-- {Definicion de >>=}
State (\s -> let (v, s') = runState m s 
              in runstate (return v) s') =                     
-- {Definicion de return}
State (\s -> let (v, s') = runState m s 
              in runstate (State (\s -> (v, s))) s') =         
-- {Definicion de runState}
State (\s -> let (v, s') = runState m s 
              in (\s -> (v, s)) s') =                          
-- {Aplicacion de funcion anonima}
State (\s -> let (v, s') = runState m s 
              in (v, s')) =                                    
-- {Sustitucion de let}
State (\s -> runState m s) =                                   
--{Eta-reduccion}
State (runState m) =                                           
-- {La composicion de State y runState es la identidad}
m

-- Monad 3
(m >>= f) >>= g =                                              
-- {Definicion de >>=}
(State (\s -> let (v, s') = runState m s 
               in runState (f v) s')) >>= g =                  
-- {Definicion de >>=}
State (\s -> let (v, s') = runState (State (\s -> let (v, s') = runState m s 
                                                   in runState (f v) s')) s 
              in runState (g v) s') =                          
-- {Definicion de runState} 
State (\s -> let (v, s') = (\s -> let (v, s') = runState m s 
                                   in runState (f v) s')) s 
              in runState (g v) s') =                          
-- {Aplicacion de funcion anonima}
State (\s -> let (v, s') = (let (v, s') = runState m s 
                             in runState (f v) s')
              in runState (g v) s') =                          
-- {Se reescribe el let de la siguiente forma:  
-- let a = (let b = c in f(b)) in g(a)
-- let b = c
--     a = f(b)
--  in g(a)}
State (\s -> let (v, s')  = runState m s 
                 (w, s'') = runState (f v) s'
             in runState (g w) s'')

 
m >>= (\x -> f x >>= g) =                                      
-- {Definicion de >>=}
State (\s -> let (v, s') = runState m s 
             in runState ((\x -> f x >>= g) v) s') =          
-- {Aplicacion de funcion anonima}
State (\s -> let (v, s') = runState m s 
             in runState (f v >>= g) s') =                    
-- {Definicion de >>=}
State (\s -> let (v, s') = runState m s 
              in runState (State (\s -> let (v, s') = runState (f v) s
                                         in runState (g v) s')) s') =   
-- {Definicion de runState}
State (\s -> let (v, s') = runState m s 
              in (\s -> let (v, s') = runState (f v) s
                         in runState (g v) s') s') =           
-- {Aplicacion de funcion anonima}
State (\s -> let (v, s') = runState m s 
              in (let (v, s') = runState (f v) s'
                   in runState (g v) s')) =                    
-- {Se reescribe el let de la siguiente forma:  
-- let a = b in (let c = f(a) in g(c))
-- let a = b
--     c = f(a)
--  in g(c)}
State (\s -> let (v, s')  = runState m s 
                 (w, s'') = runState (f v) s'
             in runState (g w) s'')
             
-- {Ambos terminos son iguales a una misma expresion}
-- {Por transitividad, (m >>= f) >>= g = m >>= (\x -> f x >>= g)}
\end{lstlisting}
	
\newpage
b) Implementación del evaluador para la mónada State:
\begin{lstlisting}
-- Evalua un programa en el estado nulo
eval :: Comm -> Env
eval p = snd (runState (evalComm p) initState)

-- Evalua un comando en un estado dado
evalComm :: MonadState m => Comm -> m ()
evalComm Skip            = return ()
evalComm (Let var ie)    = do n <- evalIntExp ie 
                              update var n
evalComm (Seq c1 c2)     = do evalComm c1 
                              evalComm c2
evalComm (Cond be c1 c2) = do b <- evalBoolExp be 
                              if b then (evalComm c1)
                                   else (evalComm c2)
evalComm (While be c)    = evalComm (Cond be (Seq c (While be c)) Skip)


-- Resuelve la evaluacion de dos expresiones enteras, aplicadas a un operador.
evalIntOp :: MonadState m => IntExp -> IntExp -> (Int -> Int -> a) -> m a
evalIntOp ie1 ie2 operator = do n1 <- evalIntExp ie1
                                n2 <- evalIntExp ie2
                                return (operator n1 n2)


-- Evalua una expresion entera, sin efectos laterales
evalIntExp :: MonadState m => IntExp -> m Int
evalIntExp (Const i)       = return i
evalIntExp (Var x)         = lookfor x
evalIntExp (UMinus ie)     = do n <- evalIntExp ie 
                                return (-n)
evalIntExp (Plus ie1 ie2)  = evalIntOp ie1 ie2 (+)
evalIntExp (Minus ie1 ie2) = evalIntOp ie1 ie2 (-)
evalIntExp (Times ie1 ie2) = evalIntOp ie1 ie2 (*)
evalIntExp (Div ie1 ie2)   = evalIntOp ie1 ie2 div

                             
-- Resuelve la evaluacion de dos expresiones booleanas, aplicadas a un operador.
evalBoolOp :: MonadState m => BoolExp -> BoolExp -> (Bool -> Bool -> a) -> m a
evalBoolOp be1 be2 operator = do n1 <- evalBoolExp be1
                                 n2 <- evalBoolExp be2
                                 return (operator n1 n2)
                                 
                                 
-- Evalua una expresion booleana, sin efectos laterales
evalBoolExp :: MonadState m => BoolExp -> m Bool
evalBoolExp BTrue         = return True
evalBoolExp BFalse        = return False
evalBoolExp (Eq ie1 ie2)  = evalIntOp ie1 ie2 (==)
evalBoolExp (Lt ie1 ie2)  = evalIntOp ie1 ie2 (<)
evalBoolExp (Gt ie1 ie2)  = evalIntOp ie1 ie2 (>)
evalBoolExp (And be1 be2) = evalBoolOp be1 be2 (&&)
evalBoolExp (Or be1 be2)  = evalBoolOp be1 be2 (||)
evalBoolExp (Not be)      = do b <- evalBoolExp be
                               return (not b)

\end{lstlisting}

\newpage
\section*{Ejercicio 2}
a) Instancia de Monad para StateError: 
\begin{lstlisting}
instance Monad StateError where
    return x = StateError (\s -> Just (x, s))
    m >>= f = StateError (\s -> case runStateError m s of
                                     Nothing -> Nothing
                                     Just (v, s') -> runStateError (f v) s')
\end{lstlisting}
\paragraph{}
b) Instancia de MonadError para StateError: 
\begin{lstlisting}
instance MonadError StateError where
    throw = StateError (\s -> Nothing)

\end{lstlisting}
\paragraph{}
c) Instancia de MonadState para StateError:
\begin{lstlisting}
instance MonadState StateError where
lookfor v = StateError (\s -> maybe (Nothing) (\u -> Just (u, s)) (lookfor' v s))
                              where lookfor' v [] = Nothing
                                    lookfor' v ((u,j):ss) | v == u    = Just j
                                                          | otherwise = lookfor' v ss
update v i = StateError (\s -> Just ((), update' v i s))
                 where update' v i [] = [(v, i)]
                       update' v i ((u, j):ss) | v == u    = (v, i):ss
                                               | otherwise = (u, j):(update' v i ss)    
\end{lstlisting}
\newpage
d) Implementación del evaluador utilizando la mónada StateError:
\begin{lstlisting}
-- Evalua un programa en el estado nulo
eval :: Comm -> Env
eval c = maybe initState snd (runStateError (evalComm c) initState)

-- Evalua un comando en un estado dado
evalComm :: (MonadState m, MonadError m) => Comm -> m ()
evalComm Skip            = return ()
evalComm (Let var ie)    = do n <- evalIntExp ie 
                              update var n
evalComm (Seq c1 c2)     = do evalComm c1 
                              evalComm c2
evalComm (Cond be c1 c2) = do b <- evalBoolExp be 
                              if b then (evalComm c1)
                                   else (evalComm c2)
evalComm (While be c)    = evalComm (Cond be (Seq c (While be c)) Skip)


-- Resuelve la evaluacion de dos expresiones enteras, aplicadas a un operador.
evalIntOp :: (MonadState m, MonadError m) => IntExp -> IntExp -> (Int -> Int -> a) 
    -> m a
evalIntOp ie1 ie2 operator = do n1 <- evalIntExp ie1
                                n2 <- evalIntExp ie2
                                return (operator n1 n2)


-- Evalua una expresion entera, sin efectos laterales
evalIntExp :: (MonadState m, MonadError m) => IntExp -> m Int
evalIntExp (Const i)       = return i
evalIntExp (Var x)         = lookfor x
evalIntExp (UMinus ie)     = do n <- evalIntExp ie 
                                return (-n)
evalIntExp (Plus ie1 ie2)  = evalIntOp ie1 ie2 (+)
evalIntExp (Minus ie1 ie2) = evalIntOp ie1 ie2 (-)
evalIntExp (Times ie1 ie2) = evalIntOp ie1 ie2 (*)
evalIntExp (Div ie1 ie2)   = do n1 <- evalIntExp ie1
                                n2 <- evalIntExp ie2
                                if n2 == 0 then throw
                                           else return (div n1 n2)


-- Resuelve la evaluacion de dos expresiones booleanas, aplicadas a un operador.
evalBoolOp :: (MonadState m, MonadError m) => BoolExp -> BoolExp -> (Bool -> Bool -> a)
    -> m a
evalBoolOp be1 be2 operator = do n1 <- evalBoolExp be1
                                 n2 <- evalBoolExp be2
                                 return (operator n1 n2)
                                 
                                 
-- Evalua una expresion booleana, sin efectos laterales
evalBoolExp :: (MonadState m, MonadError m) => BoolExp -> m Bool
evalBoolExp BTrue         = return True
evalBoolExp BFalse        = return False
evalBoolExp (Eq ie1 ie2)  = evalIntOp ie1 ie2 (==)
evalBoolExp (Lt ie1 ie2)  = evalIntOp ie1 ie2 (<)
evalBoolExp (Gt ie1 ie2)  = evalIntOp ie1 ie2 (>)
evalBoolExp (And be1 be2) = evalBoolOp be1 be2 (&&)
evalBoolExp (Or be1 be2)  = evalBoolOp be1 be2 (||)
evalBoolExp (Not be)      = do b <- evalBoolExp be
                               return (not b)

\end{lstlisting}


\newpage
\section*{Ejercicio 3}
a) Mónada propuesta: 
\begin{lstlisting}
-- Monada estado
newtype StateErrorTick a = 
        StateErrorTick {runStateErrorTick :: Env -> Maybe (a, Int, Env)}
\end{lstlisting}
\paragraph{}
b) Clase MonadTick: 
\begin{lstlisting}
class Monad m => MonadTick m where
    -- Crea un contador en 1
    tick :: m ()
\end{lstlisting}
\paragraph{}
c) Instancia de MonadTick:  
\begin{lstlisting}
instance MonadTick StateErrorTick where
    tick = StateErrorTick (\s -> Just ((), 1, s))
\end{lstlisting}
\paragraph{}
d) Instancia de MonadError: 
\begin{lstlisting}
instance MonadError StateErrorTick where
    throw = StateErrorTick (\s -> Nothing)
\end{lstlisting}
\paragraph{}
e) Instancia de MonadState: 
\begin{lstlisting}
instance MonadState StateErrorTick where
lookfor v = StateErrorTick (\s -> maybe (Nothing) (\u -> Just (u, 0, s)) (lookfor' v s))
                                  where lookfor' v [] = Nothing
                                        lookfor' v ((u,j):ss) | v == u    = Just j
                                                              | otherwise = lookfor' v ss
update v i = StateErrorTick (\s -> Just ((), 0, update' v i s))
             where update' v i [] = [(v, i)]
                   update' v i ((u, j):ss) | v == u    = (v, i):ss
                                           | otherwise = (u, j):(update' v i ss)
\end{lstlisting}
\newpage
f) Implementación del evaluador utilizando la mónada StateErrorTick:
\begin{lstlisting}
-- Evalua un programa en el estado nulo
eval :: Comm -> (Int, Env)
eval c = maybe ((-1), []) 
               (\(v, count, s) -> (count, s)) 
               (runStateErrorTick (evalComm c) initState)

-- Evalua un comando en un estado dado
evalComm :: (MonadState m, MonadError m, MonadTick m) => Comm -> m ()
evalComm Skip            = return ()
evalComm (Let var ie)    = do n <- evalIntExp ie 
                              update var n
evalComm (Seq c1 c2)     = do evalComm c1 
                              evalComm c2
evalComm (Cond be c1 c2) = do b <- evalBoolExp be 
                              if b then (evalComm c1)
                                   else (evalComm c2)
evalComm (While be c)    = evalComm (Cond be (Seq c (While be c)) Skip)


-- Resuelve la evaluacion de dos expresiones enteras, aplicadas a un operador.
evalIntOp :: (MonadState m, MonadError m, MonadTick m) => IntExp -> IntExp -> 
    (Int -> Int -> a) -> m a
evalIntOp ie1 ie2 operator = do n1 <- evalIntExp ie1
                                n2 <- evalIntExp ie2
                                return (operator n1 n2)


-- Evalua una expresion entera, sin efectos laterales.
evalIntExp :: (MonadState m, MonadError m, MonadTick m) => IntExp -> m Int
evalIntExp (Const i)       = return i
evalIntExp (Var x)         = lookfor x
evalIntExp (UMinus ie)     = do n <- evalIntExp ie 
                                return (-n)
evalIntExp (Plus ie1 ie2)  = tick >> evalIntOp ie1 ie2 (+)
evalIntExp (Minus ie1 ie2) = tick >> evalIntOp ie1 ie2 (-)
evalIntExp (Times ie1 ie2) = tick >> evalIntOp ie1 ie2 (*)
evalIntExp (Div ie1 ie2)   = do n1 <- evalIntExp ie1
                                n2 <- evalIntExp ie2
                                if n2 == 0 then throw
                                           else tick >> return (div n1 n2)

-- Resuelve la evaluacion de dos expresiones booleanas, aplicadas a un operador.
evalBoolOp :: (MonadState m, MonadError m, MonadTick m) => BoolExp -> BoolExp -> 
    (Bool -> Bool -> a) -> m a
evalBoolOp be1 be2 operator = do n1 <- evalBoolExp be1
                                 n2 <- evalBoolExp be2
                                 return (operator n1 n2)
                                 
-- Evalua una expresion booleana, sin efectos laterales
evalBoolExp :: (MonadState m, MonadError m, MonadTick m) => BoolExp -> m Bool
evalBoolExp BTrue         = return True
evalBoolExp BFalse        = return False
evalBoolExp (Eq ie1 ie2)  = evalIntOp ie1 ie2 (==)
evalBoolExp (Lt ie1 ie2)  = evalIntOp ie1 ie2 (<)
evalBoolExp (Gt ie1 ie2)  = evalIntOp ie1 ie2 (>)
evalBoolExp (And be1 be2) = evalBoolOp be1 be2 (&&)
evalBoolExp (Or be1 be2)  = evalBoolOp be1 be2 (||)
evalBoolExp (Not be)      = do b <- evalBoolExp be
                               return (not b)

\end{lstlisting}
\end{document}